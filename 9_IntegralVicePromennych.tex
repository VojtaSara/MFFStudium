
\documentclass{article}
\usepackage{amssymb}
\usepackage{amsmath}
\usepackage{ae} 
\usepackage{xltxtra}
\usepackage[margin=1in]{geometry}
\title{Integrál více proměnných}
\author{Vojtěch Šára, Marek Douša}
\begin{document}
\maketitle
....
Podobně jako jedné proměnné, děláme horní a dolní součty, pokud se rovnají, tak existuje Riemannův integrál na daném intervalu.\\
\textbf{Tvrzení} Riemannův integrál $\int_{J}f(x)dx$ existuje právě když pro $\forall \epsilon > 0$ existuje rozdělení P takové, že $S_{J}(f,P) - s_{J}(f,P) < \epsilon$.\\\\

Každá spojitá funkce $f J \rightarrow \mathbb{R}$ na n-rozměrném kopmaktním intervalu má Riemannův integrál.\\\\

Důkaz:\\
V $\mathbb{E}_{n}$ budeme užívat vzdálenost $\sigma$ definovanou jako $\sigma (x,y) = \max_{i} |x_{i} - y_{i}|$. Jelikož je $f$ stejnoměrně spojitá, můžeme pro $\epsilon > 0$ 
zvolit $\delta > 0$ takové, že:
$$\sigma (x,y) < \delta \Rightarrow |f(x)|$$

\end{document}