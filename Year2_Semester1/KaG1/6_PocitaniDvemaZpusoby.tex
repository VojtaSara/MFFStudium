\documentclass{article}
\usepackage{amssymb}
\usepackage{amsmath}
\usepackage{ae} 
\usepackage[margin=1in]{geometry}
\title{Počítání dvěma způsoby}
\author{Vojtěch Šára, Marek Douša}
\begin{document}
\maketitle
Abstract: určím nějakou veličinu dvěma způsoby, dám je dohromady a dostanu tak explicitní vzorec pro danou veličinu.\\
%TODO: doplnit příklad p. d. způsoby u Cayleho formule%
Počet hran symetrického bipartitního grafu je $(\frac{n}{2})^{2}$\\
Pozorování: grafy bez $C_{3}$ mohou mít až kvadraticky mnoho hran vzhledem k $|v|$.\\
\textbf{Věta:}\\
Pokud graf G na n vrcholech neobsahuje $C_{4}$ jako podgraf, potom má 
nejvýše $\frac{1}{2}(n^{\frac{3}{2}}+n)$ hran.\\\\
\textbf{Důkaz:}\\
Spočítáme dvěma způsoby počet vidliček $P_{3}$. Pokud zafixuji dva vrcholy, tak vidím, že vidliček je maximálně
tolik kolik je dvojic různých vrcholů - $\# vidliček \leq \binom{n}{2}$.\\ 
Zároveň můžeme vybírat dvojice sousedních vrcholů každého vrcholu - ty také tvoří vidličky a tímto způsobem
také sumou přes všechny vrcholy dojdeme k celkovému počtu vidliček.
$\# vidliček = \sum_{v \in V} \binom{deg(v)}{2}$\\
Dosazením do rovnosti je $\sum_{v \in V} \binom{deg(v)}{2} \leq \binom{n}{2}$ => $\sum_{v \in V} (def(v)-1)^2 \leq n^2$
Cauchy Schwartzova nerovnost ($<x|y> \leq ||x||*||y||$). \\
$x={deg(v1)-1,deg(v2)-1},...,deg(vn)-1$
$y={1,1,1,...,1}$
Šikovně zvolím x a y, t.ž. mi vyjde $<x|y> = \sum_{i = n} deg(i)-1 = 2*|E| - n$ \\
$||x|| = \sqrt{\sum_{i = n} deg(i)-1} \leq \sqrt{n^2} = n$ z rovnice co nám vyšla dříve
$||y|| = \sqrt{\sum_{i=n} 1^2}=\sqrt{n}$
vycházi $2*|E| - n \leq n*\sqrt{n}$, tedy že počet hran je maximálně $\frac{1}{2}(n^{3/2} + n)$ \\
$A,B \subseteq X$ nezávislé ... $A \subsetneq B \wedge B \subsetneq A$\\
Antiřetězec / nezávislá množina $Y \subseteq P(X): \forall A,B \in Y$ jsou $A,B$ navzájem nezávislé.\\\\
Spernerova Věta:\\
Každý množinový systém vybudovaný na n prvcích má nejvýše $\binom{n}{n/2}$ nezávislých podmnožin.\\
Důkaz:\\
Označíme M největší antiřetězec v $(S, \subseteq)$ kde $S \subseteq P(X), |x| = n$.
$m = \{M_{1},M_{2},...,M_{k}\}$. Dvěma způosby spočítejme $\#(M,Ř)$, kde $M \in m$ a $Ř$ je maximální
řetězec obsahující M.\\
Pozorování: $\#(M,Ř) \leq \#Ř \leq n!$ protože k jednomu Ř lze doplnit nanejvýš jedno M. Řetězců je maximálně tolik,
kolik je různých uspořádání. \\
Druhý způsob: \\
Pozorování - každou množinu M lze nalézt v nejvýše $|M|!(n-|M|)!$ řetězcích. Nakreslíme-li Ř obsahující M.\\
$$Ř é \emptyset = M_{0},M_{1},...,M_{i}=M,...,M_{n}=X$$. $\#(M,Ř) = \sum_{M \in m} |M|!(n-|M|)!$\\
$$\sum_{M \in m} |M|!(n-|M|)! \leq n!$$
$$\sum_{M \in m} \frac{|M|!(n-|M|)!}{n!} \leq -1$$
$$1 \geq \sum_{M \in m} \binom{n}{|M|}^{-1} \geq \sum_{M \in m} \binom{n}{\frac{n}{2}}^{-1} = |m| \binom{n}{\frac{n}{2}}^{-1}$$
Poslední krok: $ \Rightarrow  |m| \leq \binom{n}{\frac{n}{2}}$\\\\
Věta nám umožnuje testovat částečném uspořádání na celém grafu.

Míra souvislosti grafů\\
Můžeme převést na jednodušší úlohu podrozdělením grafu na více částí.
Definice: Řekněme, že $S \subseteq V_{0}$ je vrcholový řez (separátor) pokud G\S je nesouvislý.
Vrcholová souvislost neprázdného grafu F je $K_{v(G)} = min|S|, kde S je vrcholový řez, G\neq K_{n}$ neboli nejmenší počet vrcholů, který už graf udělá nesouvislý.
Graf je t-souvislý pokud $K_{v}(G)\geq t$
Jak se souvislost mění na podgrafech?
Pokud má stejně vrcholů, tak má podgraf $\leq$ souvislost.
Pokud má méně, tak o něm nemohu říct nic.
$K_{v}(G) \leq $min deg(n) 
Hranová souvislost $\neq$ Vrcholová souvislost.
$\#$ Hranových řezů $\geq \#$ vrcholových řezů.w
\end{document}