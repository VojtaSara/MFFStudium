\documentclass{article}
\usepackage{amssymb}
\usepackage{amsmath}
\usepackage{ae} 
\usepackage[margin=1in]{geometry}
\title{Implicitní funkce}
\author{Vojtěch Šára, Marek Douša}
\begin{document}
\maketitle
Úloha: řešit systém rovnic
$$F_{1}(x,y_{1},...,y_{n}) = 0,$$
$$... ... ...$$
$$F_{n}(x,y_{1},...,y_{n}) = 0 $$
pro $y_{i}$ jako korektně definované $f_{i}(x)$\\
A. Jedna rovnice: $F(x,y) = 0$ V okolí bodu $(x^{0},y_{0})$ (horní index není mocnina, ale index vektoru)
předpokládáme o F: spojité parciální derivace do řádu $k \geq 1$,\\
Pak se nějak použije lagrangova věta?????\\
Z toho nám vyjde spojitost a derivace F. Díky tomu můžu spočítat libovolný řád derivace v
f, z čehož získám Taylorův polynom - kterým aproximuji funkci, kterou hledám (asi teda hledáme f?).\\
B. Dvě rovnice: 
$$F_{1}(x,y_{1},y_{2}) = 0$$
$$F_{2}(x,y_{1},y_{2}) = 0$$
Řeším nějak jen jednu a druhou řeším parametrem (?) a spočítám determinant, který z
nějakého důvodu nesmí být nulový.\\
Jacobiho determinant 
$$\frac{D(F)}{D(y)} = \det(\frac{\partial F_{i}}{\partial y_{j}})_{i,j=1,...,m}$$
Připomeňme si, že determinant má geometrický význam objemu rovnoběžnostěnu. Stejně tak,
jako funkce jedné proměnné natahuje nebo stlačuje délky malých kousků okolo $x$ v poměru
absoultní hodnoty derivace v $x$, tak vektorová funkce $f = (f_{1},f_{2},...)$ natahuje nebo stlačuje objemy
malých kousků oblasti U okolo $x$ v poměru absoultní hodnoty Jakobiánu.\\\\
Věta: Buďte $F_{i}(x,y_{1},...y_{m}), i = 1,...,m$ funkce n+m proměnných se spojitými
parciálními derivacemi do řádu $k \geq 1$. Buď:
$$F(x^{0},y^{0}) = o$$
a 
$$\frac{D(F)}{D(y)}(x^{0},y^{0}) \neq 0$$
(Proč tady je Jakobián je zatím nějak tajemné?) Potom existují $\delta > 0 \Delta > 0$
takové, že pro každé 
$$x \in (x_{1}^{0}-\delta, x_{1}^{0} +\delta) \times ... \times (x_{n}^{0}-\delta, x_{n}^{0} +\delta)$$
existuje právě jedno
$$x \in (y_{1}^{0}-\delta, y_{1}^{0} +\delta) \times ... \times (y_{n}^{0}-\delta, y_{n}^{0} +\delta)$$
t.ž.: $$F(x,y) = 0$$.\\\\

\textbf{Vázané extrémy.}\\
Jestli v daném bodě není derivace nulová, pak v tomto bodě nemůže být extrém. Takto
můžeme snadno vygenerovat všechny kandidáty na extrémy (kde je derivace nulová).
Akorát si musíme dát pozor na krajní body!\\
Ve více proměnných je hledání kandidátů stejně snadné - totální diferenciál musí 
být nulový. (Musím být ale ve vnitřním bodu definičního oboru - ne na kraji)
$$\frac{\partial f}{\partial x_{i}}(a) = 0, i=1,...,n$$
Zde je ale bodů na kraji nekonečně mnoho.\\
Příklad: $f(x,y) = x+2y$ na kruhu $B = \{(x,y)|x^{2}+x^{2}\leq 1\}$. Kruh B je 
kompaktní a tedy funkce f nabývá maxima i minima na B. Parciální derivace jsou ale
kladné, tudíž nám nedávají kandidáty na extrémy. Jak to vyřešíme?\\\\

Zkusíme najít extrémy funkce $f(x_{1},...,x_{n})$ jako podřízené vazbám $g_{i}(x_{1},...,x_{n}),
i = 1,...,k$ (zde funkce g jsou funkce, které omezují f - v našem příkladu je to kruh
omezující danou funkci)\\
Věta: Buďte $f,g_{1},...g_{k}$ reálné funkce def. na $D \subseteq \mathbb{E}_{n}$; 
nechť mají spojité parciální derivace. Nechť je hodnost matice
$$M = \begin{pmatrix}
    \frac{\partial g_{1}}{\partial x_{1}}, & ... & \frac{\partial g_{1}}{\partial x_{n}}\\
    ... & ... & ...\\
    \frac{\partial g_{k}}{\partial x_{1}}, & ... & \frac{\partial g_{k}}{\partial x_{n}}
\end{pmatrix}$$
maximální, tedy $k$ v každém bodě oboru D. Jestliže f nabývá v bodě $a = (a_{1},...,a_{n})$
lokálního extrému podmíněného vazbami g. 
Pak existují čísla $\lambda_{1}, ..., \lambda_{k}$ taková, že pro každé $i=1,...,n$ platí:
$$\frac{\partial f(a)}{\partial x_{i}} + \sum_{j=1}^{k}\lambda_{j} \cdot \frac{\partial g_{j}(a)}{\partial x_{i}} = 0$$\\
Aplikujeme formuli na předchozí příklad:
$\frac{\partial f}{\partial x} = 1$ a $\frac{\partial f}{\partial y} = 2$, $g(x,y) = x^{2}+y^{2}-1$
takže $\frac{\partial g}{\partial x} = 2x$ a $\frac{\partial g}{\partial y} = 2y$ 
Máme tedy jedno $\lambda$ které splňuje dvě rovnice
$$ 1 + \lambda \cdot 2x = 0 \quad 2 + \lambda \cdot 2y = 0$$
To je možné jen když $y = 2x$. Tedy, jelikož $x^{2} + y^{2} = 1$ dostáváme $5x^{2} = 1$
a tedy $x = +-\frac{1}{\sqrt{5}}$; to lokalizuje extrémy do $(\frac{1}{\sqrt{5}}, \frac{2}{\sqrt{5}})$
a $(\frac{-1}{\sqrt{5}},\frac{-2}{\sqrt{5}})$\\
Předpokládáme existenci čísel $\lambda_{1}, ..., \lambda_{k}$ splňujích víc než k rovnic,
těmto lambdám se říká Lagrangeovy multiplikátory.\\\\

\textbf{Důkaz:}\\
Matice M má hodnost k právě když aspoň jedna její $k\times k$ podmatice je regulární
(a tedy má nenulový determinant). Dejme tomu:
$$
\begin{vmatrix}
    \frac{\partial g_{1}}{\partial x_{1}}, & ... & \frac{\partial g_{1}}{\partial x_{k}}\\
    ... & ... & ...\\
    \frac{\partial g_{k}}{\partial x_{1}}, & ... & \frac{\partial g_{k}}{\partial x_{k}}
\end{vmatrix} \neq 0
$$
Potom podle věty o implicitních funkcích máme v okolí bodu a funkce $\phi_{i}(x_{k+1},...,x_{n})$
se spojitými parciálními derivacemi takové, že (pišme $\tilde{x} = (x_{k+1},...,x_{n})$)
$$g_{i}(\phi_{1}(\tilde{x}), ..., \phi_{k}(\tilde{x}), \tilde{x}) = 0$$ pro $i = 1,...,k$
Pak přeskočím počítání z lineární algebry a dostanu to co jsem chtěl dokázat (wow) 
(jakože tu formuli na konci).\\\\

\textbf{Další využití Věty o implicitních funkcích: Regulární zobrazení}\\
Buď $U \subseteq \mathbb{E}_{n}$ otevřená a nechť mají
$$ f_{i}, i = 1, ..., n$$
spojité parciální derivace. Řekneme, že výsledné zobrazení
$$ f = (f_{1},...,f_{n}) : U \rightarrow \mathbb{E}_{n}$$
je regulární, jestliže je
$$\frac{D(f)}{D(x)}(x) \neq 0$$
pro $\forall x \in U$\\\\
Je-li $f: U \rightarrow \mathbb{E}_{n}$ regulární, je obraz $f[V]$ každé otevřené podmnožiny
$V \subseteq U$ otevřený.\\\\
Regulární zobrazení je lokálně prosté.\\\\
Prosté regulární zobrazení má regulární inverzi.
\end{document}