
\documentclass{article}
\usepackage{amssymb}
\usepackage{amsmath}
\usepackage{ae} 
\usepackage{xltxtra}
\usepackage[margin=1in]{geometry}
\title{Integrál více proměnných}
\author{Vojtěch Šára, Marek Douša}
\begin{document}
\maketitle
....
Podobně jako jedné proměnné, děláme horní a dolní součty, pokud se rovnají, tak existuje Riemannův integrál na daném intervalu.\\
\textbf{Tvrzení} Riemannův integrál $\int_{J}f(x)dx$ existuje právě když pro $\forall \epsilon > 0$ existuje rozdělení P takové, že $S_{J}(f,P) - s_{J}(f,P) < \epsilon$.\\\\

Každá spojitá funkce $f J \rightarrow \mathbb{R}$ na n-rozměrném kompaktním intervalu má Riemannův integrál.\\\\

Důkaz:\\
V $\mathbb{E}_{n}$ budeme užívat vzdálenost $\sigma$ definovanou jako $\sigma (x,y) = \max_{i} |x_{i} - y_{i}|$. Jelikož je $f$ stejnoměrně spojitá, můžeme pro $\epsilon > 0$ 
zvolit $\delta > 0$ takové, že:
$$\sigma (x,y) < \delta \Rightarrow |f(x)|$$



... Zde chybí zápisek z hodiny ~25. listopadu\\\\


Primitivní rozšíření na vyšší dimenzi - povolíme si pouze rovnoběžnostěn jako Df, pak můžeme rozdělit tvar ve všech dimenzích rozdělením $\rightarrow$ z toho máme cihličky
ve více dimenzích, které můžeme opět sčítat jako 2dimenzionální cihličky = obdélníčky. Takto získáme většinu vlastností integrálu, které už známe. Co ovšem nemáme je
protějšek základní věty Analýzy. To v následujícím textu zkusíme napravit.\\\\

Fubiniova věta\\
\textit{Pozor, x a y jsou zde vektory!}\\
Vezměme součin $J = J' \times J'' \subseteq \mathbb{E}_{m+n}$ intervalů $J' \subseteq \mathbb{E}_{m}$, $J'' \subseteq \mathbb{E}_{n}$. Nechť $\int_{J}f(x,y)dxy$ existuje a
nechť pro $\forall x \in J'$ (resp. $\forall y \in J''$) existuje $\int_{J''}f(x,y)dy$ (resp. $\int_{J'}f(x,y)dx$). Potom je
$$\int_{J}f(x,y)dxy = \int_{J'}(\int_{J''}f(x,y)dy)dx = \int_{J''}(\int_{J'}f(x,y)dx)dy $$\\

Ve třech dimenzích pro příklad tedy máme:
$$\int_{a_{1}}^{b_{1}}( \int_{a_{2}}^{b_{2}}( \int_{a_{3}}^{b_{3}}( f(x_{1},x_{2},x_{3}))dx_{3} )dx_{2} )dx_{1}$$\\

Pro nahlédnutí proč Fubini funguje:
$$\sum_{i\leq m, j \leq n}x_{ij} = \sum_{i\leq m}(\sum_{j\leq n} x_{ij})$$

Samotný důkaz:\\
Položme $F(x) = \int_{J''}f(x,y)dy$. Dokážeme, že $\int_{J'}F$ existuje a že $\int_{J}f = \int_{J'}F$. Zvolme rozdělení P intervalu J tak aby
$$\int f - \epsilon \leq s(f,P) \leq S(f,P) \leq \int f + \epsilon$$
Toto P je tvořeno rozděleními P' intervalu J' a P'' intervalu J''. Máme 
$$\mathcal{B}(P) = \{ B' \times B'' | B' \in \mathcal{B}(P'), B'' \in \mathcal{B}(P'')\}$$
Každá cihla P se objeví jako právě jedno $B' \times B''$. Potom je

$$F(x) \leq \sum_{B'' \in \mathcal{B}(P'') \max_{y \in B''} f(x,y) \times \text{vol}(B'')}$$
a tedy
$$S(F,P') \leq \sum_{B' \in \mathcal{P'}} \max_{x \in B'}(\sum_{B'' \in \mathcal{B}(P'')}\max_{y \in B''}f(x,y) \cdot \text{vol}(B'')) \cdot \text{vol}(B') \Rightarrow \dots$$
To už se nějak dopočítá s nehlédnutím o sumách, které je těsně před důkazem.\\\\

Příklad: Objem koule.\\
Interval $J = \langle -r, r \rangle \times \langle -r, r \rangle $ vezměme:
$$f(x,y) = \begin{cases}
    \sqrt{r^{2} - x^{2} - y^{2}} \text{ pro }  r^{2}-x^{2}-y^{2} \geq 0
    0 \text{jinak}
\end{cases}$$
Podle Fubiniovy věty
$$2 \int_{-r}^{r}(\int_{-r}^{r}f(x,y)dy)dx = 2 \int_{-r}^{r}(\int_{-u}^{u}f(x,y)dy)dx$$ kde $u = \sqrt{r^{2} - x^{2}}$. Pak už jen spočítáme...\\\\


Co uděláme, pokud máme divoký definiční obor? Uvědomme si, že při užití Fubiniovy věty dostaneme více integrálů jedné proměnné, které už mají konečně mnoho bodů nespojitosti a to nám nevadí.
Takováto argumentace ale není dostatečná - netuším proč, ale intuice docela sedí.\\\\

První poznámka o Lebesgueově integrálu.\\
Riemannův integrál se dá rozšířit tak, že např. můžeme korektně počítat
$$\int  \lim f_{n} = \lim \int f_{n}$$
předpokládajíce jen, že $|f_{n}| \leq K$.\\
To je přesně Lebesgeův integrál - rozšíření Riemannova integrálu, které funguje lépe v některých speciálních případech.\\\\

Tietzova věta:\\
Mějme reálnou spojitou omezenou funkci na uzavřené množině. Pak tato funkce lze rozšířit na celý prostor\dots \\\\

S těmito dvěma věcmi bez důkazu vyřešíme náš problém s ošklivými definičními obory. Tady už jsem moc nevěděl, ale v příští přednášce by se toto mělo zlepšit - téma bude Lebesgeův integrál.
(TOTO NEBUDE U ZKOUŠKY ANYWAYS)

\end{document}