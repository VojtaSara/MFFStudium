\documentclass{article}
\usepackage{amssymb}
\usepackage{amsmath}
\usepackage{ae} 
\usepackage{tikz}
\usepackage[margin=1in]{geometry}
\title{Knapsack pomocí evoluce}
\author{Vojtěch Šára}
\begin{document}
\maketitle
Jedince jsem kódoval pomocí indikátorových vektorů - tedy jedinec 1, 0, 1 odpovídá tomu, že první a třetí předmět se do batohu
vyberou a prostřední ne. Díky tomu je spočítání fitness jedno maticové násobení.\\
Selekci jsem vybral ruletovou, protože mi experimentálně vyšla trochu lepší.\\
Mutace je náhodný bitflip, nic jiného ani s použitým kódováním nepřipadá v úvahu, nějaké permutování by zcela vymazalo informaci o jedinci.\\
Mutaci jsem nastavil po vyzkoušení více hodnot na 0.8, protože mi experimentálně vyšla
takto trochu agresivnější mutace lépe, jinak se simulace příliš zasekává v latentním prostoru
na nějakém lokálním minimu, agresivnější mutace pomáhá z těchto lokálních minim vyšťouchávat.
\end{document}