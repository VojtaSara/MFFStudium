\documentclass{article}
\usepackage{amssymb}
\usepackage{amsmath}
\usepackage{ae} 
\usepackage[margin=1in]{geometry}
\title{Neprocedurální programování 1}
\author{Vojtěch Šára}
\begin{document}
\maketitle
Program v prologu se skládá z klauzulí. Funktory + atomy = termy, s tečkou na konci z toho je klauzule.
\begin{center}
    muz(adam).\\
    muz(kain).\\
    muz(abel).\\
    zena(eva).\\
    rodic(adam,kain).\\
    rodic(adam,abel).\\
    rodic(eva,abel).\\
    rodic(eva,kain).\\
\end{center}
Toto definuje například predikát muz() - ten je definovaný právě na těch prvcích, které vyjmenujeme. Pak se můžeme
ptát na dotazy a zkompilovaný program prologu nám odpoví true nebo false. Nebo můžeme zadat dotaz "muz(X)"
což nám vypíše všechny hodnoty, pro které platí, že jsou muž.\\
Více dotazů můžeme spojit čárkou (= konjunkcí).\\\\

Chceme přidat predikát otec. To uděláme pomocí pravidla:\\
otec(Kdo, Dite) :- rodic(Kdo, Dite), muz(Kdo)\\
Toto se při dotazu "Otec(adam,kain)" vyhodnocuje rekurzivně - deklarovali jsme cíl, ten se rozdělí na 
podcíle, splnění všech podcílů indikuje úspěch - splnění cíle. \\\\\

\subsection*{Procedura s více pravidly}
clovek(C) :- zena(C).
clovek(C) :- muz(C).
! klauzule jsou procházeny v pořadí, ve kterém jsou napsány. To, co jsme napsali, je skrytá disjunkce.
Jinak disjunkci mohu napsat pomocí středníku.\\\\

bratr(Bratr,Osoba):- rodic(R,Bratr),rodic(R,Osoba),muz(Bratr). Nedefinuje bratra korektně! Potřebujeme,
aby Bratr a Osoba byly různé. Na to potřebujeme negaci, ta se píše backslash =. \\\\

Anonymní proměnná - $\_$ na názvu této proměnní nezáleží - nechci jí použít nikdy znovu.\\\\

% Druhá přednáška
term - (jednoduchý - (konstanta - atom / číslo / []) / proměnná) / (složený)\\
složený - muz(adam) = funktor s jednoduchým nebo složeným termem, rekurzivní definice\\\\

Program se skládá z procedur, procedura je posloupnost klauzulí.\\
Fakt = pravidlo s prázdným tělem. \\
Direktiva = pravidlo s prázdnou hlavou, např :- consult(demo). toto je command pro překladač.\\
\subsubsection*{Procedurální versus neprocedurální}
Proc.: proměnné - deklarovány, lokální / globální, změna hodnoty přiřazením\\
Neproc.: proměnné - dynamická alokace paměti - GC, platnost proměnné je omezena na klauzuli, v níž se vyskytuje.
Proměnná volná / vázaná.\\
Proc.: datový typ záznam - položky identifikovány jménem.\\
Neproc.: složený term - položky identifikovány polohou, stromová struktura.\\\\

\subsubsection*{Řešení Einsteinovy hádanky}
Strategie - použijeme složené termy.\\
kolej(\_,\_,pokoj(\_,\_,\_,skyrim,\_),pokoj(gates,\_,\_,\_,\_),pokoj(\_,brno,\_,\_,\_)) - toto je dobrý první krok, další neuvádíme.\\\\

\subsubsection*{Syntax - Operátory}
Proc.: výrazy - a+b\\
Neproc.: adam \= eva\\

Unifikace - dva terma lze unifikovat, pokud jsou identické, nebo se stanou identickými po substituci vhodné hodnoty proměnným v obou 
termech.\\
Unifikační algoritmus - ten je vyvolán =. Algoritmus se pokusí najít substituci, která unifikuje.

\subsubsection*{Peano aritmetika v prologu}
prirozene\_cislo(X) :- X je přirozené číslo.\\
prirozene\_cislo(0). \\
prirozene\_cislo(s(X)) - s je succesor\\\\

mensi(X,Y):- X je ostře menší než Y\\
mensi(0,s(X)) :- prirozene\_cislo(X).\\
mensi(s(X),s(Y)) :- mensi(X,Y).\\\\

\subsubsection*{Strukturální rekurze}
Řízena strukturou argumentů - rekurzivní datová struktura. Složená z báze a kroku rekurze, realizovaným rekurzivním pravidlem.\\
soucet(0,X,X) :- prirozene\_cislo(X).\\
soucet(s(X),Y,s(Z)) :- soucet(X,Y,Z).\\\\

\subsubsection*{Seznamy}
Seznam realizovaný pomocí relace. Zápis [a,b,c].






\end{document}