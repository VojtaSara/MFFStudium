\documentclass{article}
\usepackage{amssymb}
\usepackage{amsmath}
\usepackage{ae} 
\usepackage[margin=1in]{geometry}
\title{Úvod do AI - 2. přednáška}
\author{Vojtěch Šára}
\begin{document}
\maketitle
Reflex agent - pouze reagují na momentální situaci. Mohu ho modelovat funkcí - obzervace $\rightarrow$ akce. Model-based reflex agent:
(PastState, PastAction, Observation) $\rightarrow$ state - zde state je v podstatě primitivní paměť.\\
Goal-based agent, stejný jako model-based, ale navíc bere v potaz reward function. Tento agent umí hledat a plánovat ve stavovém prostoru.

Stavy jsou\\
Atomické - nedělitelné\\
Faktorované - stav je vektor\\
----- - -----\\\\

Problem solving agent - má nějaké cílové stavy, akce jsou přechody mezi stavy, úkol je najít sekvenci akcí, která skončí v cílovém
stavu. Hledání ve stavovém prostoru.\\



\end{document}