\documentclass{article}
\usepackage{amssymb}
\usepackage{amsmath}
\usepackage{ae} 
\usepackage[margin=1in]{geometry}
\title{3. Dú z pravděpobodobnosti a statistiky}
\author{Vojtěch Šára}
\begin{document}
\maketitle
a) Spočítám podle vzorečku pro střední hodnotu:
$$1(\frac{1}{12})+ 2(\frac{2}{12} + 4(\frac{4}{12}) + 5(\frac{5}{12})) = \frac{23}{6}$$
b) Použiji následující vzoreček pro spočítání rozptylu:
$$var(x) = \frac{1}{n}\sum_{i = 1}^{n}(x_{i}-\mathbb{E}[X]^{2})p_{i} = \frac{1}{12}((1-\frac{23}{6}^{2})1 + (2-\frac{23}{6}^{2})2 + (4-\frac{23}{6}^{2})4 + (5-\frac{23}{6}^{2})5) = $$
$$ = \frac{65}{3}\frac{1}{12} = \frac{65}{36} \simeq 1.8$$
c) Odchylka je tedy odmocnina rozptylu, tedy: $\sqrt{1.8} \simeq 1.343$\\\\


Výsledky experimentu v R:\\
> kostka = (sample(c(1,2,2,4,4,4,4,5,5,5,5,5),1000,replace=T))\\
> mean(kostka)\\
$[1] 3.928$\\
> var(kostka)\\
$[1] 1.694511$\\
> sd(kostka)\\
$[1] 1.301734$\\
\end{document}