\documentclass{article}
\usepackage{amssymb}
\usepackage{amsmath}
\usepackage{ae} 
\usepackage[margin=1in]{geometry}
\title{Druhý domácí úkol z Pravděpodobnosti a statistiky}
\author{Vojtěch Šára}
\begin{document}
\maketitle
Použiji Bayesovu větu ke spočítání P(Vybral jsem cinknutou minci | padl šestrkát orel).
$$P(A|B) = \frac{P(B|A)P(A)}{P(B|A)P(A)+P(B|A^{c})P(A^{c})}$$
$$P(A|B) = \frac{1\cdot \frac{1}{100}}{1\cdot \frac{1}{100}+\frac{1}{2}^{6}\frac{99}{100}} = 0.39$$
Pravděpodobnost, že jsme si vytáhli dvouorlovou minci je tedy 0.39.
\end{document}