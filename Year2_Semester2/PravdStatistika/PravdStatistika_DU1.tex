\documentclass{article}
\usepackage{amssymb}
\usepackage{amsmath}
\usepackage{rotating}
\usepackage{longtable}
\usepackage{ae} 
\usepackage[margin=1in]{geometry}
\title{1. Dú z pravděpobodobnosti a statistiky}
\author{Vojtěch Šára}
\begin{document}
\maketitle
% \usepackage{longtable}
\begin{sidewaystable}
    


\begin{longtable}{|l|l|l|l|l|l|l|l|l|l|l|l|l|l|l|l|l|l|l|l|l|l|l|l|l|l|l|l|l|l|l|l|l|l|l|l|} 
    \hline
    1 &   &   &   &   &   & 2 &   &   &   &   &   & 3 &   &   &   &   &   & 4 &   &   &   &   &   & 5 &   &   &   &   &   & 6 &   &   &   &   &    \endfirsthead 
    \hline
    1 & 2 & 3 & 4 & 5 & 6 & 1 & 2 & 3 & 4 & 5 & 6 & 1 & 2 & 3 & 4 & 5 & 6 & 1 & 2 & 3 & 4 & 5 & 6 & 1 & 2 & 3 & 4 & 5 & 6 & 1 & 2 & 3 & 4 & 5 & 6  \\ 
    \hline
      &   &   &   &   &   &   &   &   &   &   &   &   &   &   &   &   &   &   &   &   &   &   & X &   &   &   &   & X &   &   &   &   & X &   &    \\ 
    \hline
      &   &   &   &   &   &   &   &   &   &   &   &   &   &   &   &   &   &   &   &   &   &   &   &   &   &   &   &   &   & X & X & X & X & X & X  \\ 
    \hline
      &   &   &   &   & X &   &   &   &   &   & X &   &   &   &   &   & X &   &   &   &   &   & X &   &   &   &   &   & X &   &   &   &   &   & X  \\
    \hline
    \end{longtable}

\end{sidewaystable}

Vytvořil jsem si tabulku všech možností (na následující straně) - první řádek odpovídá tomu, co padlo na první kostce,
druhý řádek tomu co na druhé. Následující tři řádky odpovídají třem jevům:\\
a) Součet = 10\\
b) Na první padlo 6\\
c) Na jedné z kostek padlo 6\\
Pravděpodobnosti snadno vyčtu z tabulky, vždy kolik příslušný řádek obsahuje "X" děleno všemy jevy.
(Počítám s tím, že kostka je férová a pravděpobodobnostní prostor je tedy klasický).\\
$$P(\textbf{a)}) = \frac{1}{12} $$
$$P(\textbf{b)}) = \frac{1}{6} $$
$$P(\textbf{c)}) = \frac{11}{36} $$
Podmíněné pravděpodobnosti také snadno vyčtu z tabulky s pomocí vzorečku:
$$P(A | B) = \frac{P(A \cap B)}{P(B)}$$
Kde pravděpodobnost průniku je počet X které sdílí dva dané řádky děleno 36.\\
$$P(\textbf{a) | b)}) = \frac{1}{6}$$
$$P(\textbf{a) | c)}) = \frac{2}{11}$$
$$P(\textbf{b) | a)}) = \frac{1}{3}$$
$$P(\textbf{b) | c)}) = \frac{6}{11}$$
$$P(\textbf{c) | a)}) = \frac{2}{3}$$
$$P(\textbf{c) | b)}) = \frac{1}{1}$$
\end{document}