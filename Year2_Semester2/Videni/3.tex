\documentclass{article}
\usepackage{amssymb}
\usepackage{amsmath}
\usepackage{ae} 
\usepackage{xltxtra}
\usepackage{graphicx}
\graphicspath{ {.} }
\usepackage[margin=1in]{geometry}
\title{1. DÚ z počítačového vidění - cvičení 3}
\author{Vojtěch Šára}
\begin{document}
\maketitle
\subsubsection*{1}

Zaprvné chci dokázat, že minimalizací následujícího:
$$\frac{1}{n}\sum_{i}^{n}d^{2}(x_{i},Q(x_{i}))$$
splňuji podmínku nejbližšího souseda. To je snadné, neb minimalizací sumy určitě minimalizuji i všechny její členy. Pokud by tedy pro spor nějaký člen
měl blíže k jiné barvě palety, než která mu byla přidělena funkcí Q (tím by porušoval podmínku nejbl. souseda), pak přiřadit této konkrétní barvě tu bližší 
barvu palety by zmenšilo velikost sumy, což je ve sporu s předopokladem, že velikost sumy byla minimální.\\\\

\subsubsection*{2}
Nyní ukážu že minimalizací
$$\frac{1}{n}\sum_{i}^{n}d^{2}(x_{i},Q(x_{i}))$$
splňuji i podmínku centroidu. Sumu si mohu rozdělit na sumy jednotlivých regionů:
$$\frac{1}{|R_{0}|}\sum_{\{i | x_{i} \in R_{0}\}}^{|R_{0}|}d^{2}(x_{i},c_{0}) +
\frac{1}{|R_{1}|}\sum_{\{i | x_{i} \in R_{1}\}}^{|R_{1}|}d^{2}(x_{i},c_{1}) \dots
\frac{1}{|R_{k}|}\sum_{\{i | x_{i} \in R_{k}\}}^{|R_{k}|}d^{2}(x_{i},c_{k})
$$
Minimalizací celé této sumy minimalizuji i její části, tedy jednotlivé sumy, obecně tedy následující je minimalizováno:
$$\frac{1}{|R_{k}|}\sum_{\{i | x_{i} \in R_{k}\}}^{|R_{k}|}d^{2}(x_{i},c_{k})$$
Jelikož vzdálenost každých dvou barev je větší než jedna tak mohu vypustit mocninu (minimalizací mocniny minimalizuji i číslo před umocněním, pokud je $geq 1$):
$$\frac{1}{|R_{k}|}\sum_{\{i | x_{i} \in R_{k}\}}^{|R_{k}|}d(x_{i},c_{k})$$
Z toho již lze odvodit, že pokud tento konkrétní člen má být minimální, pak i aritmetický průměr přes $x_{i}$ v každém regionu bude jeho těžištěm.

\end{document}
