\documentclass{article}
\usepackage{amssymb}
\usepackage{amsmath}
\usepackage{ae} 
\usepackage[margin=1in]{geometry}
\title{Automaty a gramatiky 1}
\author{Vojtěch Šára}
\begin{document}
\maketitle
Nejjednodušší automat je vypínač. 
Deterministický konečný automat (DFA) $A = (Q,\Sigma, \delta, q_{0},F)$\\
Q - konečná neprázdná množina stavů\\
$\Sigma$ - konečná neprázdná množina vstupních symbolů abecedy\\
$\delta$ - přechodová funkce $Q \times \Sigma \rightarrow Q$ typicky reprezentovaná hranami grafu\\
$q_{0} \in Q$ - počáteční stav, vede do něj šipka odnikud.\\
F - množina koncových stavů\\\\

Pro představu si můžeme představit vyhledávací automat z Knuth Morris Pratta.\\\\

Automat mohu reprezentovat diagramem, tabulkou či stavovým stromem.\\

Slovo je konečná (i prázdná) posloupnost symbolů z $\Sigma$, prázdné slovo se značí $\lambda \vee \epsilon$
Množina všech slov v abecedě $\Sigma$ značíme $\Sigma^{\star}$\\
Množina neprázdných slov je $\Sigma^{+}$\\
Jazyk je množina slov v abecedě $\Sigma$ - $L \subseteq \Sigma^{\star}$\\\\

Zřetězení - $u.v$ nebo $uv$\\
Mocnina (jen přirozenými) - ($u^{0} = \lambda, u^{1} = u, u^{n+1} = u^{n}.u$)\\
Délka slova $|auto| = 4$\\
Počet výskytů je kolikrát je podřetězec v řetězci.\\

Rozšířená přechodová funkce $\delta^{\star}$ je transitivní uzávěr funkce $\delta$\\\\

Jazyk je rozpoznatelný konečným automatem jestliže existuje konečný automat A takový,
že  $L = L(A) = \{w | w \in \Sigma^{\star} \wedge \delta^{\star}(q_{0},w) \in F\}$\\\\

Jazyky rozpoznatelné konečnými automaty nazýváme regulární.\\\\

Jsou jazyky, které nejsou regulární. \\\\

\subsection{Iterační (pumping) lemma}
Mějme regulární jazyk L. Pak existuje $n \in \mathbb{N}$ závislá na L, taková, že:
každé $w \in L; |w| \geq n$ můžeme rozděliz na tři části, w = xyz, že:\\
\begin{enumerate}
    \item $y \neq \lambda$
    \item |xy| $\leq n$
    \item $\forall k \in \mathbb{N}_{0}$ slovo $xy^{k}z$ je také v L.
\end{enumerate}
Důkaz:\\
Regulární - existuje DFA A s n stavy, že L = L(A).\\
Vezměme libovolné slovo z L délky $m \geq n$.\\
Zbytek důkazu neni.\\\\

% Druhá přednáška

\subsubsection*{Kongruence}
Na konečné abecedě $\Sigma$ a relaci ekvivalence $\sim$ na $\Sigma^{\star}$. Potom:
\begin{enumerate}
    \item $\sim$ je pravá kongruence, jestliže $(\forall u,v,w \in \Sigma^{\star})u \sim v \Rightarrow uw \sim vw$
    \item $\sim$ je konečného indexu, jestliže rozklad $\Sigma^{\star} / \sim$ má konečný počet tříd.
    \item Třídy kongruence = $[u]_{\sim}$
\end{enumerate}

\subsubsection*{Myhill-Nerodova věta}
Nechť L je jazyk nad konečnou abecedou $\Sigma$. Potom následující je ekvivalentní:
\begin{enumerate}
    \item L je rozpoznatelný konečným automatem
    \item existuje pravá kongruence konečného indexu tak, že L je sjednocením jistých tříd rozkladu $\Sigma^{\star} / \sim$
\end{enumerate}
Pro důkaz definuji ekvivalenci jako slova, která nás z počátečního stavu dostanou do stejného stavu. Tato ekvivalence
je ekvivalencí, je to i pravá kongruence.\\
Pro druhý směr volíme stavy jako třídy rozkladu $\Sigma^{\star} / \sim$\\
Počáteční stav $q_{0}$...\dots\\\\

Věta Myhil-Nerod je konstruktivní, nabízí nám tedy návod jak sestavit automat. Na přednášce tady byl příklad.\\\\

Zajímavost: jazyk $L = \{u|u = a^{+}b^{i}c^{i} \vee u = b^{i}c^{j}\}$ není regulární (Myhill-Nerod) ale vždy lze pumpovat
první písmeno.

\subsubsection*{Dosažitelnost}
Stav je dosažitelný, pokud existuje slovo, kterým se do něj z počátečního stavu dostanu. Dosažitelné stavy
hledáme iterativně - nějaký průchod stavovým prostorem. Korektnost - nachází pouze dosažitelné stavy, 
Úplnost - všechny dosažitelné stavy najdu.\\\\

\subsubsection*{Nejednoznačnost}
Automat přijímající daný jazyk není určen jednoznačně. Tady byl na přednášce příklad takového jazyka.

\subsubsection*{Automatový homomorfismus}
$h : Q_{1} \rightarrow Q_{2}$ je automat. homomorfismus, jestliže:\\
\begin{enumerate}
    \item $h(q_{0_{1}}) = q_{0_{2}}$
    \item $h(\delta_{1}(q,x)) = \delta_{2}(h(q),x)$stejné přechodové funkce
    \item stejné koncové stavy
\end{enumerate}

\subsubsection*{Ekvivalence automatů}
Dva konečné automaty jsou ekvivalentní, pokud rozpoznávají stejný jazyk. Homomorfismus implikuje ekvivalenci.\\
Jestli jsou všechny stavy ekvivalentní, pak jsou automaty ekvivalentní. Neekvivalentní = rozlišitelné.\\\\

Existuje algoritmus hledání všech rozlišitelných stavů. Postupně vyškrtávám nehodící. Tento algoritmus je korektní.\\
Algoritmus nalezení reduktu - vyškrtnu nedosažitelné, pak spojím ekvivalentní stavy. Tím dostávám redukovaný DFA.\\

\end{document}