\documentclass{article}
\usepackage{amssymb}
\usepackage{amsmath}
\usepackage{ae} 
\usepackage[margin=1in]{geometry}
\title{Druhý domácí úkol z Automatů a Gramatik}
\author{Vojtěch Šára}
\begin{document}
\maketitle
Libovolnou metodou dokažte, že jazyk $L=\left\{w\left|w \in\{1\}^{*},\right| w \mid=p, p\right.$ je prvočíslo $\}$ není regulární.\\\\
Tvrzení dokážu obměnou pumping lemmatu. Konkrétněji tedy ukážu, že pro jakkoli zvolené n se mohu vypumpovat z jazyka ven.
Začnu volbou prvočísla, které je větší než $n$, označme ho $p$. Pak mohu vzít slovo $w = 1^{p}$. Slovo $w$ rozložíme na $w=xyz$, délku
prostřední části - $|y|$ označíme písmenem $a$. $|xz| = p - a$. Pumping lemma nám říká, že pumpováním tohoto slova zůstaneme v jazyce,
konkrétně například slovo $xy^{p-a}z$ by mělo být v jazyce. Když si ale rozepíšeme, co by to znamenalo:
$$|xy^{p-a}z| = |xz| + |y^{p-a}| = p-a + |y|(p-a) = (p-a)(1+|y|) = (p-a)(1+a)$$
Jelikož bylo $a$ větší než nula, tak nám tato rovnice říká, že $|xy^{p-a}z|$ je nutně složené číslo, protože je násobkem dvou nějakých
čísel.
\end{document}